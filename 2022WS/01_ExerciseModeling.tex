\documentclass[
        a4paper,
        pdftex,
        english, 
        oneside,%twoside, 
        listof=totoc,%liststotoc, 
        bibliography=totoc, %bibtotoc,
        titlepage,
        abstracton 
]{scrartcl} %scrartcl

\usepackage[
    left=2.5cm,right=2.5cm,top=2cm,bottom=2cm,
    includehead, includefoot
]{geometry}

%\clubpenalty = 9999
%\widowpenalty = 9999 
%\displaywidowpenalty = 9999


%%%%%%%%%%%%%%%%%%%%%
%package definitions
%%%%%%%%%%%%%%%%%%%%%
\usepackage[latin1]{inputenc} 

\usepackage{graphicx}
\usepackage{amsmath}
\usepackage{amssymb}
\usepackage{amsthm}
\usepackage{subfigure}
\usepackage{url}
\usepackage{multirow}
\usepackage[normalem]{ulem}
\usepackage{color}
\usepackage[np,autolanguage]{numprint}
\usepackage{nicefrac}
\usepackage{units}
\usepackage{tocbasic}
\usepackage{rotating}
\usepackage{floatflt}

\usepackage{caption}
\captionsetup{
    %format=hang,labelfont=bf,font={small,sf},
    skip=7pt
}


\usepackage{float}

\usepackage[pdftex]{hyperref}
\hypersetup{
    unicode=false,          % non-Latin characters in Acrobat’s bookmarks
    pdftoolbar=true,        % show Acrobat’s toolbar?
    pdfmenubar=true,        % show Acrobat’s menu?
    pdffitwindow=true,      % window fit to page when opened
    pdfstartview={FitV},    % fits the width of the page to the window
    pdftitle={Data Management: Exercise 01 -- Data Modeling},    % title
    pdfauthor={Matthias Boehm}, % author
    pdfsubject={},   % subject of the document
    pdfcreator={},   % creator of the document
    pdfproducer={}, % producer of the document
    pdfkeywords={}, % list of keywords
    pdfnewwindow=true,      % links in new window
    bookmarksnumbered=true, % put section numbers in bookmarks
    bookmarksopen=true,     % open up bookmark tree
    bookmarksopenlevel=1,   % \maxdimen level to which bookmarks are open
    colorlinks=false,        % false: boxed links; true: colored links
    linkcolor=black,        % color of internal links  
    citecolor=blue,         % color of links to bibliography
    filecolor=black,        % color of file links
    urlcolor=black          % color of external links
}

%\makeindex{notation}
\makeindex

\definecolor{gray}{RGB}{235,235,235}

% Specific environments
\newenvironment{itemize*}{\begin{itemize}\setlength{\itemsep}{4pt}\setlength{\parskip}{0pt}\setlength{\parsep}{0pt}}{\end{itemize}}
\newenvironment{enumerate*}{\begin{enumerate}\setlength{\itemsep}{4pt}\setlength{\parskip}{0pt}\setlength{\parsep}{0pt}}{\end{enumerate}}   


\begin{document}	

\pagenumbering{arabic}
\frenchspacing

%%%%%%%%%%%%%%%%%%%%%%%%%%%%%%%%%%%%%%%%%%%%%%%%%%%%%%%%%%%%%%%%%%%%%%%%%%%%%%%%%%%%%%%%%%%%%%%%%%%%%%%%%%%%%%%%%%%%%%%%

\begin{flushleft}
\noindent \textbf{Hussain Hussain, Adrian Spataru}\\
\textit{Based on previous exercises from Prof. Matthias B\"ohm} \\
Graz University of Technology\\
Computer Science and Biomedical Engineering\\
Institute of Interactive Systems and Data Science\\
\end{flushleft}

\bigskip

\setcounter{section}{0}
\section{Data Management WS22/23: Exercise 01 -- Data Modeling}
\noindent \textbf{Published: October 10th, 2022}\\
\noindent \textbf{Deadline: October 30th, 2022, 11.59pm}\\

%motivation%
\noindent This exercise on data modeling aims to provide practical experience in Entity-Relationship (ER) modeling, ER-relational mapping, and relational normalization. The expected result is a zip archive named \texttt{DBExercise01\_<student\_ID>.zip}, containing the partial results of the individual sub-tasks, submitted in TeachCenter.

\subsection{ER Modeling (15/25 points)}
\label{sec:ER}

Create an ER diagram in Modified Chen (MC) notation---including entity types, relationship types, attributes, cardinalities, and keys (create surrogate keys if natural unique identifiers are missing)---for managing a music social network\footnote{Dataset link: \url{https://github.com/tug-db/datasets/tree/main/LFM-1b/raw}. A cleaned version of the dataset will be made available for Exercise 2 after additional data preparation and cleaning.}.
Please use a digital tool for designing the ER diagram. We strongly recommend draw.io\footnote{\url{https://app.diagrams.net/}} since it has a toolbar for ER elements.
There are multiple correct ways of modeling (alternatives, and level of details), however the diagram should capture the essential information of the following discourse:
\begin{itemize}
\item The platform contains \textit{users} who register their age, gender and country. A user also has a timestamp (date and time) of their registration. Each user can form friendships with other users.
\item The dataset contains several \textit{artists}, described by their names. Each user can listen to different artists on the platform. We register the user-artist playcount, that is the number of times a user listened to an artist on the platform.
\item The artist plays one or more \textit{genres} of music.
\item The platform also keeps record of upcoming music \textit{festivals}. Each festival takes place in a certain country on a certain date and lasts for one or more days. A festival can either be indoors or outdoors, and may have a known standard ticket price. A festival features one or more genres.
\item Each user can buy a number of tickets to one or more festivals. The set of tickets bought by a user can be either standard or VIP tickets.
\item Each \textit{country} is described by its name and a unique country code (which consists of 3 letters) in addition to its population, and GDP (gross domestic product) per capita.
\end{itemize}

\noindent \textbf{Partial Result:} \texttt{ERDiagram.pdf}

\subsection{Mapping ER Diagrams into the Relational Model (10/25 points)}
\label{sec:RM}

Create a relational schema for the ER diagram designed in Task~\ref{sec:ER} and bring it into third normal form, assuming functional dependencies from country codes to their names.
This schema should include the relations and typed attributes, as well as all primary and foreign keys.
% Furthermore, the schema should also ensure that no user is friends with themselves (TODO no duplicates).
% Furthermore, the schema should also ensure that no person can vote multiple times at a single election (e.g., via in-person and mail-by-ballot voting).
It is up to you if you provide either a SQL script (\texttt{CreateSchema.sql}) with \texttt{CREATE TABLE} statements, or provide a text schema (\texttt{Schema.txt}) in the following text notation (one line per table):
\begin{verbatim}
<Table>(<Attribute 1>:<type>(PK), <Attribute 2>:<type>, ..., <Attribute n>(FK))
\end{verbatim}
Here, PK and FK indicate primary and foreign keys, where multiple attributes with (PK) represent a composite primary key. If an attribute is both foreign key and (part of) a primary key, use \texttt{<Attribute m>(PK,FK)}. Please, adhere to this notation with unchanged parentheses and delimiters, and limit yourself to common data types (i.e., \texttt{int}, \texttt{numeric(p,s)}, \texttt{char(n)}, \texttt{varchar(n)}, or \texttt{date}) because this sub-task is subject to automated grading.\\

\noindent \textbf{Partial Result:} \texttt{Schema.txt}, or \texttt{CreateSchema.sql}

\bigskip

\end{document}